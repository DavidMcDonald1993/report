\documentclass[•]{article}

\usepackage{amsmath}
\begin{document}

\section{Cut}

%minimising the number of edges cut \cite{kernighan1970efficient,fiduccia1982linear,ding2001min}
%
%this only cut small subgraphs, so ration cut \cite{wei1991ratio}, normalised cut \cite{shi2000normalized} and min-max cut \cite{ding2001min}
%
%
%Metis \cite{karypis1998fast} - MQI \cite{gallo1989fast,lang2004flow} algorithm 
%metis - splits graph into two equal sized pieces 
%MQI - low
%
%
%\cite{van2001graph}
%markov clustering algorithm (MCL)
%two phases to simulate diffusion process on graph - expansion/inflation
%only k largest elements are kept at each phase for efficiency

\section{modularity}

%\cite{girvan2002community}
%seminal
%hierarchical divisive algorithm
%removed edges based on their betweeness until modularity quality function is maximised
%
%\begin{align*}
%Q - \sum_{i=1}^k (e_{ii} - a_i^2)
%\end{align*}
%where $e_i$ is the probability that edge $i$ is in module $i$ and $a_i$ is the probability that a random edge would be in module $i$

%\cite{clauset2004finding}
%sped up \cite{newman2005power}
%links are iteratively added to produce the largest increase in modularity
%
%\cite{guimera2005functional}
%same as above but with simulated annealing

%\cite{blondel2008fast}
%multistage algorithm
%local optimisation of GN modularity on neighbourhood of each node
%communities are replaced by a single node
%repeat until modularity stops increasing

%\cite{radicchi2004defining}
%dividing hierarchical like GN
%does not use edge betweenness - uses edge clustering co-efficient based on loops and stops based on quality of communities not modularity
%strong community - internal < external degree
%week community - total internal < total external degree




%\cite{rosvall2007information}
%translates the problem of clustering to the problem of optimally compressing the information on the graph so that the most information is uncovered when the compression is decoded
%uses SA to minimise a function that represents compression and data loss
%slow
%
%\cite{rosvall2008maps}
%same principle as above but with a dynamic process
%used SA to optimise a compression based on minimum length of a random walk



%\cite{newman2007mixture}
%uses bayesian inference to deduce best model to fit data represented by graph structure
%advantage: can find best group structure (not necessarily communities)
%disadvantage: k must be known a-priori

%\cite{fortunato2007resolution}
%modularity optimisation can lead to incorrect community divisions due to a `resolution limit' (fails to identify modules smaller than a scale that depends on the size of the network)


%\cite{ronhovde2010local}
%overcomes the resolution limit of modularity optimisation by not comparing to a null model but instead penalising communities for missing edges 



%
%
%\cite{yang2013hierarchical}
%proposes Probabalistically Mining Communities (PMC)
%two phases - 
%heuristic - random walk to reduce search space for communities
%optimisation - constrained quadratic objective function
%integrates advantages of both and offers a trade-off between effectiveness and efficiency

%\cite{leskovec2010empirical}
%conductance:
%\begin{align*}
%\phi(S) = \frac{c_s}{\min(Vol(S),Vol(V \setminus S)} 
%\end{align*}
%with
%\begin{align*}
%c_s = |\{(u,v) : u \in S, v \not\in S\}|
%\end{align*}
%
%best conductance cut 
%
%modularity for bipartite networks: \cite{barber2007modularity}
%
%
%\cite{andersen2006local}
%local parititoning algorithm using a variation of PageRank
%the ordering of vertices produced by PageRank produces a cut with small conductance

%\section{real-time}
%
%\cite{leung2009towards}
%analyse `label propagation/epidemic' a technique proposed in \cite{raghavan2007near}

\section{spectral-based}

spectral clustering

\cite{donetti2004detecting}
based on spectral components of graph
select g eigen vectors and project points to g-dimensional space
group with traditional hierarchgical clustering techiniques and maximise modularity


%\section{overlapping}

%%\cite{adamcsek2006cfinder}
%%cfinder
%%first algorithm to consider overlapping communities
%%rolls k-cliques across networks 
%%computationally expensive -- time increases exponentially with graph size
%
%
%%\cite{palla2005uncovering}
%%clique percolation method (CPM)
%%overlapping large scale networks
%%communities are unions of k-cliques
%
%\cite{shen2009detect} 
%proposed EAGLE to detect hierarchical and overlapping communities on the word association and scientific collaboration networks

%\cite{lancichinetti2009detecting}
%first algorithm for both hierarchical and overlapping communities
%local algorithm to optimise 
%\begin{align*}
%f_G = \frac{k_{in}^G}{(k_{in}^G + k_{out}^G)^\alpha}
%\end{align*}
%
%\cite{psorakis2011overlapping}
%bayesian non-negative matrix factorisation 
%`soft-partitioning' and assigns node participation scores to modules


%\cite{reichardt2006statistical}
%potts
%finding ground state of spin
%can find hierarchies and overlap
%`communities structure interpreted as the spin configuration that minimises energy of spin glass'
%uses modularity of random graphs as null model 
%
%\cite{ronhovde2009multiresolution}
%potts method
%minimise hamiltonian of potts-like spin model
%resolution parameter for scale of communities to detect


%\cite{pons2011post}
%two contributions:
%multi-scale quality functions that can uncover hieracrhical communities and produce several differnent partitions.
%post-processing of clusters found by hierarchical methods (encoded in a dendogram)
%
%detailed overview: \cite{xie2013overlapping}

\cite{balcan2013modeling}
formalised tight communities and hierarchical structures

\section{EA}

locus-based adjacency \cite{park1998genetic}

\cite{handl2007evolutionary}
proposed MOCK a multi objective EA for clustering with automatic k-determination
two objective functions - connectedness and deviation
two phases - clustering using adapted PESA-II and an encoded that does not require k a priori
model selection that selected best clustering based on synthetic unclustered data generated from a poisson distribution

\cite{pizzuti2008ga}
defined quality function based on power mean
changing power $r$ changes the weight of densely inter-connected nodes

\section{Neural networks}

\cite{defferrard2016convolutional}
generalised CNNS for graph input


\cite{kipf2016semi}
variant of CNNS designed specifically for graphs
semi-supervised (some nodes are labelled)
encodes both graph structure and node features


\section{SOM}

%seminal paper \cite{kohonen1990self} based on earlier work about topological ordering \cite{kohonen1982self}
%later work by kohonen -%- LVQ \cite{kohonen1995learning}

%\cite{kiang2001extending}
%clustering is under researched
%successful for classification:
%medical diagnosis \cite{vercauteren1990classification}
%image and character recognition \cite{sabourin1993modeling,delbimbo1993three}
%speech recognition \cite{leinonen1993self}

%\cite{kiang2001extending} clusters SOM output using contiguity constrained clustering
%also \cite{vesanto2000clustering} clusters output of SOM
%-compared k-means and hierarchical
%performs well against clustering alone with less computation time


%\cite{fritzke1994growing}
%unsupervised and supervised (RBF) SOM for clustering
%automatically finds network structure and size
%controlled growth and removal


\cite{yamakawa2006self}
som with graph input
constrained weights to point to edges on a graph
shortest path metric is used
\begin{align*}
w = (n_i, n_j, \beta)
\end{align*}


\bibliography{references}
\bibliographystyle{unsrt}

\end{document}